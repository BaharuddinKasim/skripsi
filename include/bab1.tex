\chapter{PENDAHULUAN}

\section{Latar Belakang}

Pesatnya perkembangan teknologi dalam bidang informasi dan komunikasi saat ini membawa pengaruh terhadap seluruh kegiatan yang dilakukan oleh organisasi maupun pemerintahan. Semakin tinggi teknologi yang digunakan akan semakin mempercepat pula proses penyampaian informasi. Berbagai alat komunikasi yang canggih seperti telepon, seluler, telegram, faksimile dan lain sebagainya. Namun masih ada komunikasi tertulis yang tidak dapat dilupakan keberadaannya, bahkan sampai sekarang masih tetap terpakai seolah tak bisa tergantikan oleh berbagai peralatan komunikasi yang canggih. Komunikasi tertulis tersebut adalah surat. Namun masih banyak yang ditemukan dalam suatu instansi atau perusahaan yang melakukan berbagai kesalahan dalam mengelola surat atau data-data penting yang ada. Seperti ditemukannya data atau surat yang tercecer ataupun rusak \thecite{Vironica}. Proses pertukaran informasi yang cepat dan tepat dapat memperlancar kegiatan administrasi dalam pemerintahan, khususnya dalam kegiatan administrasi yang berkaitan dengan hal surat menyurat. Salah satu contoh kegiatan administrasi surat menyurat di Kantor Desa Lampenai yang kurang efisien karena masih menerapkan sistem yang manual.

Desa Lampenai merupakan organisasi pemerintahan di bawah camat yang terletak di Kecamatan Wotu Kabupaten Luwu Timur. Desa Lampenai ini terletak di sebelah barat Kabupaten Luwu Timur yang memiliki wilayah terluas yaitu 22,31 km$^2$ atau meliputi 17\% dari luas kecamatan. Selama ini proses administrasi surat menyurat di Desa Lampenai masih didata dengan cara mencatat ke dalam buku besar. Dengan memisahkan proses surat masuk dan proses surat keluar kemudian menyimpan surat ke dalam map sebagai media penyimpanan surat. Dengan sistem yang masih manual dapat memungkinkan surat akan hilang, rusak dan menyulitkan dalam melakukan pencarian jika suatu waktu diperlukan. Dengan adanya masalah diatas, maka dibutuhkan suatu sistem informasi yang dapat membantu dalam hal surat menyurat serta perngarsipan surat masuk dan keluar.

Sistem Informasi Desa adalah bagian tak terpisahkan dalam implementasi Undang – Undang Desa. UU Desa Pasal 86 tentang Sistem Informasi Pembangunan Desa dan Pembangunan Kawasan Perdesaan jelas disebutkan bahwa Desa berhak mendapatkan akses informasi melalui sistem informasi yang dikembangkan oleh Pemerintah Daerah Kabupaten atau Kota (Pasal 86 UU Desa Nomor 1 tahun 2014).

Banyaknya surat yang dibuat dan diterima oleh Desa, maka pencarian data menjadi tidak efisien dalam hal waktu dan tenaga. Beberapa kekurangan dari sistem yang manual ini adalah surat tidak dapat tersimpan dengan baik karena mudah sobek dan dikhawatirkan surat yang ada sebelumnya akan hilang serta buku besar yang digunakan untuk mencatat tanggal dan jenis surat mudah rusak karena hampir setiap hari digunakan.

Sistem Informasi Surat Menyurat pada Desa merupakan sebuah program yang diolah untuk mempermudah dalam manajemen surat masuk atau keluar. Dengan adanya fitur arsip surat, manajemen surat akan semakin lebih baik dan mempermudah pada saat pencarian data dari surat masuk dan surat keluar, dengan memasukkan tanggal dan jenis surat kita sudah bisa melihat data-data surat yang kita inginkan.

Untuk mengatasi hal tersebut, maka dibutuhkan suatu Sistem Informasi Manajemen Persuratan yang lebih terintegrasi. Aplikasi ini menggunakan sistem komputerisasi meliputi pembuatan surat keluar dan pengarsipan surat masuk dan surat keluar. Oleh karena itu, dengan dibangunnya suatu aplikasi ini diharapkan dapat memperbaiki sistem yang terdahulu dan dapat mempermudah pekerjaan sehingga dapat menyingkat waktu agar efisiensi kerja mengalami peningkatan serta memudahkan pekerja dalam melakukan pengoperasiannya.

\section{Rumusan Masalah}

Berdasarkan uraian pada latar belakang masalah di atas, maka dapat dirumuskan bahwa masalah yang melatarbelakangi pada penelitian ini adalah kurang efisiennya penginputan data surat masuk dan surat keluar yang masih manual ke dalam buku besar serta pencarian data surat yang juga masih manual. Maka rumusan masalah pada penelitian ini yaitu:

\begin{enumerate}[topsep=0pt,itemsep=0pt,partopsep=0pt, parsep=0pt]
    \item Bagaimana merancang dan membangun Aplikasi Manajemen Persuratan Berbasis \textit{Website} agar dapat digunakan dan membantu menyelesaikan masalah yang ada?
    \item Bagaimana mengintegrasikan Aplikasi Manajemen Persuratan Berbasis \textit{Website} dengan sistem \textit{database}?
\end{enumerate}

\section{Batasan Masalah}

Batasan masalah dilakukan agar penelitian ini dapat memberikan pemahaman yang terarah dan sesuai dengan yang diharapkan. Agar pembahasan tidak menyimpang dari pokok perumusan masalah yang ada. Adapun batasan masalah pada penelitian ini yaitu:

\begin{enumerate}[topsep=0pt,itemsep=0pt,partopsep=0pt, parsep=0pt]
    \item Pembuatan \textit{database} sistem Aplikasi Manajemen Persuratan menggunakan \textit{MySQL}.
    \item Penginputan data surat masuk dan surat keluar diakses oleh admin, yaitu dilakukan oleh pihak administrasi kantor Desa Lampenai.
    \item Sistem Informasi Manajemen Persuratan ini hanya dapat diakses atau digunakan oleh pihak administrasi kantor Desa Lampenai saja.
\end{enumerate}

\section{Tujuan Penelitian}

Berdasarkan uraian latar belakang dan rumusan masalah di atas, maka tujuan dari penelitian ini adalah:

\begin{enumerate}[topsep=0pt,itemsep=0pt,partopsep=0pt, parsep=0pt]
    \item Dapat merancang dan membangun Aplikasi Manajemen Persuratan Berbasis \textit{Website}.
    \item Mampu mengintegrasikan Aplikasi Manajemen Persuratan Berbasis \textit{Website} dengan sistem \textit{database}.
\end{enumerate}

\section{Manfaat Penelitian}

Adapun manfaat dari penelitian ini adalah:

\begin{enumerate}[topsep=0pt,itemsep=0pt,partopsep=0pt, parsep=0pt]
    \item Dapat menambah pengetahuan dalam pembuatan Sistem Informasi berbentuk \textit{Website}.
    \item Dapat menambah pengetahuan dalam pengolahan \textit{database} dengan menggunakan \textit{MySQL}.
    \item Mempermudah pekerjaan sehingga dapat menyingkat waktu agar efisiensi kerja mengalami peningkatan.
\end{enumerate}