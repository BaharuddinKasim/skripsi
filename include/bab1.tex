\chapter{PENDAHULUAN}

\section{Latar Belakang}

Pesatnya perkembangan teknologi dalam bidang informasi dan komunikasi saat ini membawa pengaruh terhadap seluruh kegiatan yang dilakukan oleh organisasi maupun pemerintahan. Semakin tinggi teknologi yang digunakan akan semakin mempercepat pula proses penyampaian informasi. Berbagai alat komunikasi yang canggih seperti telepon, seluler, telegram, faksimile dan lain sebagainya. Namun masih ada komunikasi tertulis yang tidak dapat dilupakan keberadaannya, bahkan sampai sekarang masih tetap terpakai seolah tak bisa tergantikan oleh berbagai peralatan komunikasi yang canggih. Komunikasi tertulis tersebut adalah surat. Namun masih banyak yang ditemukan dalam suatu instansi atau perusahaan yang melakukan berbagai kesalahan dalam mengelola surat atau data-data penting yang ada. Seperti ditemukannya data atau surat yang tercecer ataupun rusak \thecite{Vironica}. Proses pertukaran informasi yang cepat dan tepat dapat memperlancar kegiatan administrasi dalam pemerintahan, khususnya dalam kegiatan administrasi yang berkaitan dengan hal surat menyurat. Salah satu contoh kegiatan administrasi surat menyurat di Kantor Desa Lampenai yang kurang efisien karena masih menerapkan sistem yang manual.

Desa Lampenai merupakan organisasi pemerintahan di bawah camat yang terletak di Kecamatan Wotu Kabupaten Luwu Timur. Desa Lampenai ini terletak di sebelah barat Kabupaten Luwu Timur yang memiliki wilayah terluas yaitu 22,31 km$^2$ atau meliputi 17 persen dari luas kecamatan. Selama ini proses administrasi surat menyurat di Desa Lampenai masih didata dengan cara mencatat ke dalam buku besar. Dengan memisahkan proses surat masuk dan proses surat keluar dan menyimpan surat ke dalam map sebagai media pengarsipan surat masuk maupun keluar. Dengan sistem yang masih manual dapat memungkinkan surat akan hilang, rusak dan menyulitkan dalam melakukan pencarian jika suatu waktu diperlukan. Dengan adanya masalah diatas, maka dibutuhkan suatu sistem informasi yang dapat membantu dalam hal surat menyurat serta perngarsipan surat masuk dan keluar \thecite{Mia}.

Sistem Informasi Desa adalah bagian tak terpisahkan dalam implementasi Undang – Undang Desa. UU Desa Pasal 86 tentang Sistem Informasi Pembangunan Desa dan Pembangunan Kawasan Perdesaan jelas disebutkan bahwa desa berhak mendapatkan akses informasi melalui sistem informasi yang dikembangkan oleh Pemerintah Daerah Kabupaten atau Kota (Undang-Undang Desa Pasal 86 Tentang Sistem Informasi Pembangunan Desa dan Pembangunan Kawasan Perdesaan).

\section{Rumusan Masalah}

Berdasarkan uraian pada latar belakang masalah di atas, dapat dikemukakan pertanyaan penelitian sebagai berikut:

\begin{enumerate}[topsep=0pt,itemsep=0pt,partopsep=0pt, parsep=0pt]
    \item Bagaimana merancang dan membangun alat pendeteksi dan filtrasi air menggunakan biji kelor berbasis IoT?
    \item Bagaimana mengkalibrasi setiap sensor sehingga data yang dihasilkan cocok dengan alat aslinya?
    \item Bagaimana cara menampilkan nilai kadar pH air dan nilai kadar kekeruhan air pada LCD?
    \item Bagaimana menganalisis kinerja alat pendeteksi dan filtrasi air menggunakan biji kelor berbasis IoT?
\end{enumerate}

\section{Batasan Masalah}

Batasan masalah pada penelitian ini adalah:

\begin{enumerate}[topsep=0pt,itemsep=0pt,partopsep=0pt, parsep=0pt]
    \item Alat yang dibuat hanya dalam skala kecil (portable).
    \item Tidak menganalisis lebih lanjut tentang kandungan pada biji kelor.
    \item Alat yang dibuat hanya untuk air sumur/tanah yang keruh.
\end{enumerate}